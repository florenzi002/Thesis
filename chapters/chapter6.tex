% !TEX root = ../main.tex
\chapter{Conclusions and further developments}
In this thesis has been presented a proof of concept of a novel framework, for the deployment of interactive analytics in the context of Smart Buildings, that is based on semantic technologies. It has been pointed out the need of integration for enabling the transition from standard BMSs to SmartBMSs. The proposed solution, called KITT, is a layered, cloud based service that allows for the creation, storing and management of semantic models of a given building. It has been demonstrated how this semantic layer can help improving the quality of diagnosis and enabling more user friendly interface towards the system, achieving a higher grade of integration.
Even though this approach has been validated and the proof of concept was happily acknowledged by professionals in the field, the project is still in its infancy and far from being mature. Being as new as it is there are lot of different directions to take in order to expand and improve the approach. Some of these expansion directions have already been highlighted in the previous chapters. Let sum up the ones that, me being the author, I think are the most interesting ones:
\begin{itemize}
  \item Reasoner optimization; the proposed approach needed an ad-hoc reasoner that could go over the limitations that other reasoners imposed. The current KITT reasoner is far from being optimal, as seen in . One of the main issue is that it has been implemented, in the most naive way, as a simple full search in the space of the instances. No caching or heuristic is involved during the resolution of the rules. Studying whether classical optimization algorithm for production rules engine, like RETE, or novel ones yet to discover can improve the performances is one of the key challanges to tackle in the future.
  \item Ontology expansion; the ontology used in the project has been trimmed so that only essential relationships and concepts are left. This reduce both the complexity of the reasoning tasks and the development phase. It can happen that these restrictions can lead to difficulties in the deplyment of certain services. Studying the expressiveness of the ontology in the current status with respect to other well known ontology, like RDF-S, could help to find the optimal balance between expressivness and ease of use.
  \item Dynamic behaviour of the model; as the semantic model that sit at the center of the approach is the ``digital twin'' of the specific building it models, its status changes in time as the real building evolves or even faster. Sensors can be changed, sometimes added, sometimes removed and the same goes for locations and equipments too. On the other hand it can occur that the physical model is too coarse and it need to be replaced by a more detailed one. Every change means that the knowledge base can lose consistency and thus arise the need to think of ways to solve conflicts and update the model. It would be interesting to know if it is possible to repair the model instead of recomputing it from scratch with the new information as a ground truth.
\end{itemize}
as stated before these are just few of possible development directions that this project can take.
