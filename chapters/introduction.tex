% !TEX root = ../main.tex
\chapter*{Introduction}
\label{ch:introduction}
Industrialised countries are facing challanges that are strictly related to energy consumption. Although great efforts in studying novel energy sources are profused by both public and private entities, these solutions need to be sustained by efforts aimed at the reduction of energy wasting. Around 32\% of total energy consumption in industrialized countries is used for electricity, heating, ventilation and air-conditiong (HVAC) in commercial buildings. Studies have shown that early detection of faults in those systems and their operation could lead to energy savings between 15\% to 32\%, improving the overall buildings efficiency. This scenario does not only offer great opportunities related to environmental protection but gives companies a chance for sensible expense reduction. Around 14\% of commercial buildings in the US (as of 2012) deployed Building Management System (BMS), supported by the increasing number of available sensors and improvement in the Internet of Things technologies. Still those data are juste barely used for simple analytics due to the lack of common schema for data coming from a variety of building, vendor and location specific sources that prevent the developers to easily integrate and process those data in a more detailed way. Energy Management Systems (EMS) are usually monolithic and poorly integrated solutions tailored for specific buildings, usually these solutions are costly in terms of both money and time. They requires efforts to encode expert knowledge that can't be resued. 
