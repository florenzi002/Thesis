% !TEX root = main.tex
\chapter*{Sommario}
Smart Building è uno degli ultimi temi che si sono imposti nell'ambito della cultura ``smart'' iniziata con l'avvento dell'Internet of Things (IOT). Quando si parla di Smart Building non ci si riferisce ad una singola tecnologia ma ad un insieme di fattori ed obiettivi che mirano a evolvere il concetto di edificio inteso nel senso tradizionale del termine. Uno Smart Building è un sistema cyber-fisico che è identificato come l'insieme dei suoi elementi strutturali, dei dispositivi accessori installati, della rete di sensori e dei suoi occupanti. L'edificio assume quindi i connotati di un organismo che, in quanto tale, deve adattarsi all'ambiente, prendere decisioni e minimizzare gli sprechi di energia. Una delle applicazioni fondamentali per la realizzazione di un sistema Smart è quella di autodiagnosi, ovvero la capacità di un sistema di riconoscere situazioni di anomalia e la capacità di individuare, con un un certo grado di confidenza, la causa di tali anomalie. Applicazioni allo stato dell'arte sono costruite ad-hoc per singoli edifici e necessitano di grandi sforzi economici per essere messe in funzione. L'aggiunta di un layer semantico consente di ridurre la necessità di intervento di operatori umani altamente specializzati e automatizza parte di questi sforzi. Inoltre aggiungere una semantica fornisce informazioni di contesto che le applicazioni possono sfruttare per migliorare l'accuratezza dei propri servizi. Il lavoro svolto si è concentrato sullo sviluppo di un framework che consente la creazione automatica di tale modello semantico.
